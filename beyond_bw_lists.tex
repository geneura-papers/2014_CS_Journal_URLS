\documentclass[preprint]{elsarticle}
\biboptions{round, numbers}
\usepackage[latin1]{inputenc}
%\usepackage[T1]{fontenc}
%\usepackage{textcomp}
\usepackage{graphicx}
%\usepackage{color}
%\usepackage{setspace}
\usepackage{url}
\usepackage[english]{babel}

\begin{document}

\begin{frontmatter}

%%%%%%%%%%%%%%%%%%%%%%%%%%%%%%%   TITLE   %%%%%%%%%%%%%%%%%%%%%%%%%%%%%%%

\title{[TENTATIVE] Can I navigate this web? Controlling URL accesses in the enterprise by means of categorical classifiers}

%%%%%%%%%%%%%%%%%%%%%%%%%%%%%%%   AUTHORS   %%%%%%%%%%%%%%%%%%%%%%%%%%%%%%%

\author{P. de las Cuevas, A.M. Mora, J.J. Merelo}
\ead{\{paloma, amorag, jmerelo\}@geneura.ugr.es}
\address{Departamento de Arquitectura y Tecnolog�a de Computadores.\\ ETSIIT - CITIC. University of Granada, Spain}
%\author{A. M. Mora}
%\ead{amorag@geneura.ugr.es}
%\address{Departamento de Arquitectura y Tecnolog�a de Computadores. Escuela T�cnica Superior de Ingenier�as Inform�tica y de Telecomunicaci�n. CITIC. University of Granada, Spain}

%\maketitle

%
%%%%%%%%%%%%%%%%%%%%%%%%%%%%%%%%%   ABSTRACT   %%%%%%%%%%%%%%%%%%%%%%%%%%%%%%%%%
%
\begin{abstract} 

\end{abstract}

%
%%%%%%%%%%%%%%%%%%%%%%%%%%%%%%%%%   KEYWORDS   %%%%%%%%%%%%%%%%%%%%%%%%%%%%%%%%%
%
\begin{keyword}
Black List \sep White Lists \sep Data Mining \sep Corporate Security Policies \sep URL request\sep Machine Learning \sep Classification.
\end{keyword}

\end{frontmatter}


%-------------------------------------------------------------------------------
%%%%%%%%%%%%%%%%%%%%%%%%%%%%%%%   INTRODUCTION   %%%%%%%%%%%%%%%%%%%%%%%%%%%%%%%
%-------------------------------------------------------------------------------

\section{Introduction}
\label{sec:intro}


%----------------------------------------------------------------------------
%%%%%%%%%%%%%%%%%%%%%%%   BACKGROUND AND RELATED WORK %%%%%%%%%%%%%%%%%%%%%%%
%----------------------------------------------------------------------------

\section{Background and related work}
\label{sec:background_sota}

%*** Esta secci�n presentar� conceptos de seguridad en la empresa, as� como conceptos de machine learning y sus aplicaciones dentro de la seguridad corporativa, en concreto en aproximaciones similares a esta
%Creo que no tiene que ver y es repetir lo mismo all over again

*** Hay que diferenciar este trabajo de los existentes y destacar su avance en el estado del arte

%Network and internet security in enterprises
%URL classification
%Crime data mining and forensics in enterprise (?)
%feature extraction

%---------------------------------------------------------------------
%\subsection{Corporate Security}
%\label{subsec:corporatesecurity}

%---------------------------------------------------------------------
%\subsection{Machine Learning}
%\label{subsec:dm+ml}



%-------------------------------------------------------------------------------
%%%%%%%%%%%%%%%%%%%%%%%%%%%%%  PROBLEM DEFINITION  %%%%%%%%%%%%%%%%%%%%%%%%%%%%%
%-------------------------------------------------------------------------------

\section{Problem definition}
\label{sec:problem_data}

*** Definir el problema a resolver y describir los datos que se van a manejar en el mismo.

%----------------------------------------------------------------------------
%%%%%%%%%%%%%%%%%%%%%%%%%%%%%%%   METODOLOGY  %%%%%%%%%%%%%%%%%%%%%%%%%%%%%%%
%----------------------------------------------------------------------------

\section{Metodology}
\label{sec:metodology}

*** Describir la metodolog�a a seguir para la creaci�n de los juegos de datos:
- preprocesado
- etiquetado para componer datos a clasificar
- t�cnicas de balanceo

*** Describir los 3 grandes bloques y justificarlos:
- Datos iniciales
- Caracter�sticas extra�das
- Sesiones

%---------------------------------------------------------------------
\subsection{Initial Data}
\label{subsec:initial_data}

*** Describir los primeros datos y procesamientos (eliminar duplicados, agrupar redundantes, etc)


%---------------------------------------------------------------------
\subsection{Extracting Features}
\label{subsec:extracting_features}

*** Describir y justificar las caracter�sticas extra�das de los datos iniciales.

%---------------------------------------------------------------------
\subsection{Grouping in Sessions}
\label{subsec:sessions}

*** Describir y justificar el proceso de agrupamiento en sesiones



%----------------------------------------------------------------------------
%%%%%%%%%%%%%%%%%%%%%%%%%   EXPERIMENTS AND RESULTS  %%%%%%%%%%%%%%%%%%%%%%%%
%----------------------------------------------------------------------------

\section{Experiments and Results}
\label{sec:experiments}

*** Describir los experimentos realizados, en cada bloque.
*** Analizar los resultados obtenidos en cada bloque

%---------------------------------------------------------------------
\subsection{Initial Data Results}
\label{subsec:initial_data_results}

*** Experimentos y resultados sobre el conjunto original y los conjuntos 'refinados' (TLD, sin duplicados, etc)


%---------------------------------------------------------------------
\subsection{Extracted Features Results}
\label{subsec:extracted_features_results}

*** Experimentos y resultados sobre el conjunto con las nuevas caracter�sticas


%---------------------------------------------------------------------
\subsection{Sessions Results}
\label{subsec:sessions_results}

*** Experimentos y resultados sobre el conjunto agrupando por sesiones



%----------------------------------------------------------------------------
%%%%%%%%%%%%%%%%%%%%%%%%%%%%%%%   DISCUSSION  %%%%%%%%%%%%%%%%%%%%%%%%%%%%%%%
%----------------------------------------------------------------------------

\section{Discussion}
\label{sec:discussion}

*** Comentar los resultados de cada m�todo y analizar las ventajas e inconvenientes de cada uno, as� como las mejoras que se hayan conseguido con la extracci�n de caracter�sticas y el agrupamiento por sesiones


%----------------------------------------------------------------------------
%%%%%%%%%%%%%%%%%%%%%%%%%%%%%%%   CONCLUSIONS  %%%%%%%%%%%%%%%%%%%%%%%%%%%%%%%
%----------------------------------------------------------------------------

\section{Conclusions and Future Work}
\label{sec:conclusions}



%%%%%%%%%%%%%%%%%%%%%%%%%%%%%  ACKNOWLEDGEMENTS %%%%%%%%%%%%%%%%%%%%%%%%%%%%%%%%

\section{Acknowledgements}
This work has been supported by the European project MUSES (FP7-318508).

\bibliographystyle{elsarticle-num}
\bibliography{review_muses,data_mining_urls,ci_security_rules}

\end{document}
