\documentclass[preprint]{elsarticle}
\biboptions{round, numbers}
\usepackage[latin1]{inputenc}
%\usepackage[T1]{fontenc}
%\usepackage{textcomp}
\usepackage{graphicx}
%\usepackage{color}
%\usepackage{setspace}
\usepackage{url}
\usepackage[english]{babel}
\usepackage{todonotes}

\begin{document}

\begin{frontmatter}

%%%%%%%%%%%%%%%%%%%%%%%%%%%%%%%   TITLE   %%%%%%%%%%%%%%%%%%%%%%%%%%%%%%%

\title{[TENTATIVE] Should I navigate this web? Controlling access in the enterprise using categorical classifiers}
%No se deben usar acr�nimos en el t�tulo - JJ
%%%%%%%%%%%%%%%%%%%%%%%%%%%%%%%   AUTHORS   %%%%%%%%%%%%%%%%%%%%%%%%%%%%%%%

\author{P. de las Cuevas, A.M. Mora, J.J. Merelo}
\ead{\{paloma, amorag, jmerelo\}@geneura.ugr.es}
\address{Departamento de Arquitectura y Tecnolog�a de Computadores.\\ ETSIIT - CITIC. University of Granada, Spain}
%\author{A. M. Mora}
%\ead{amorag@geneura.ugr.es}
%\address{Departamento de Arquitectura y Tecnolog�a de Computadores. Escuela T�cnica Superior de Ingenier�as Inform�tica y de Telecomunicaci�n. CITIC. University of Granada, Spain}

%\maketitle

%
%%%%%%%%%%%%%%%%%%%%%%%%%%%%%%%%%   ABSTRACT   %%%%%%%%%%%%%%%%%%%%%%%%%%%%%%%%%
%
\begin{abstract} 

Many companies, concerned about the safety of the connections their employees establish to different web services, implement different security techniques. 
Some reduce security risks maintaining blacklists and whitelists of web services that employees might use.
In the case of this technique, it is of great importance to determine whether an URL is dangerous or not.
In this work we propose using categorical classifiers to classify connections to be added to the blacklist or the whitelist. These models give as a result a set of rules for classifying unknown URLs.
Obtained results show that \textbf{COMENTAR LOS RESULTADOS OBTENIDOS}. %revisar que el resumen sea correcto (objetivos) y completar comentando en una frase los resultados obtenidos.  [Pedro]

\end{abstract}

%
%%%%%%%%%%%%%%%%%%%%%%%%%%%%%%%%%   KEYWORDS   %%%%%%%%%%%%%%%%%%%%%%%%%%%%%%%%%
%
\begin{keyword}
Black List \sep White Lists \sep Data Mining \sep Corporate Security Policies \sep URL request\sep Machine Learning \sep Classification.
\end{keyword}

\end{frontmatter}


%-------------------------------------------------------------------------------
%%%%%%%%%%%%%%%%%%%%%%%%%%%%%%%   INTRODUCTION   %%%%%%%%%%%%%%%%%%%%%%%%%%%%%%%
%-------------------------------------------------------------------------------

\section{Introduction}
\label{sec:intro}

The concept of Security inside an enterprise can be addressed from the
point of view of the Internet connections that are daily being
made. The employees who make these might have working purposes or
not. The enterprise, which is aware of this situation, wants to reduce
the risks of security attacks introduced by some non trustworthy
websites. Then, on a second, but still important level, companies want
to keep their employees productive and focused. % No acabo de entender
                                % el uso de may�sculas - JJ
Above all the security tools that enterprises normally use, in this
paper we want to focus on blacklists and whitelists. %Sigo sin
                                %entenderlo. Si es una definici�n,
                                %cursivas - JJ
 Their use is widely extended, as it is easy to maintain them, 
                                % easy to maintain? - JJ
                                % (Paloma) s�, es correcto, pero lo cambio si no se entiende bien
                                % Si se dice que se usan mucho, deber�a ponerse alguna referencia  [Pedro]
and each one has its advantages and disadvantages. The use of both highly increases the security in a company.
Indeed it is true that both blacklists and whitelists are easy to update or to maintain, because every day, new bad sites are reported. But at the same time, new sites (dangerous or not) are created. Netcraft reports from November \cite{netcraft:site} show that there are about 950 million active websites. But McAfee reported \cite{mcafee:site} that, at the end of the first quarter of 2014, there were more than 18 million new suspect URLs (2 million associated domains), and also more than 250 thousand new pishing URL (almost 150 thousand associated domains).

With this scenario, companies need to be able to update their blacklists and whitelists not only by what is reported by security servers but also by learning from the connections that the employees made in the past and were known as dangerous. This way, anomalous situations can be detected and also classified as candidates for the blacklist or the whitelist.

% �y si se introduce este p�rrafo como sigue?  [Pedro]
% In this work, using categorical classifiers to classify web sites is proposed.
This work shows that this can be achieved by using categorical classifiers. 
It is possible to gather information from the log files that the companies proxies store, and also from the set of Company Security Policies (CSPs), in order to be able of label the entries of the log and obtain a data set. That data set has been the one used for training and testing a set of chosen classifiers, being all of them capable of support categorical data given the nature of the information in a URL connection log file. We will show how the classifiers have obtained rules, for classifying unknown URLs, that are built with more antecedents than the URL only, as normal Black and White lists do. % yo los pondr�a en min�sculas [Pedro]

The paper is structured as follows: related work about enterprise network security, how Data Mining and Machine Learning techniques help to maintain it, and about URL classification methods, is discussed in Section 2. Section 3 provides a deeper description of the problem we aim to solve, as well as the used data and its composition. Then, after presenting the different operations performed over the initial data in Section 4, Section 5 describes the different sets of conducted experiments, in which the resulted datasets from Section 4 are tested with different classifiers. Results of those experiments are addressed in Section 6. Finally, in Section 7 we look over the conclusions of this work and give an overview about how to further continue our research in the future.
%----------------------------------------------------------------------------
%%%%%%%%%%%%%%%%%%%%%%%   BACKGROUND AND RELATED WORK %%%%%%%%%%%%%%%%%%%%%%%
%----------------------------------------------------------------------------

\section{Background and related work}
\label{sec:background_sota}

%*** Esta secci�n presentar� conceptos de seguridad en la empresa, as� como conceptos de machine learning y sus aplicaciones dentro de la seguridad corporativa, en concreto en aproximaciones similares a esta
%Creo que no tiene que ver y es repetir lo mismo all over again

*** Hay que diferenciar este trabajo de los existentes y destacar su avance en el estado del arte

%Network and internet security in enterprises


%Data

The datasets that have been traditionally used for this kind of studies are the original KDD 99 Cup Dataset \cite{kddcup99}, and its improved version NSL KDD \cite{nslkdd}. Both are detailed by Tavallaee, Mahbod et al. \cite{tavallaee2009detailed}, and the first constists of a group of connection logs from a normal network, with simulated attack traffic added. The second version, which may be called as improved or refined, had the same log of connections from the initial dataset, but having removed the redundant traffic. This means that there might have been duplicated records in the original dataset that could mislead the results after the training and testing processes.

In our work we make a similar approach. As will be deeply explained in Section \ref{sec:methodology}, we also removed the duplicated entries from the connections log. However, the main advantage of the data we have used with respect to the others used in previous works, is that we obtained the log directly from an actual company, thus we are working with more accurate datasets than those which include simulated traffic.

%URL classification

Also, in this work we are dedicating an important part to URL, its different parts and the way they influence in the classification process. We have reviewed some works related with the study of URLs. In particular, we found interesting those works that try to identify malicious sites (like the ones that want to perform a pishing attack). It is also interesting if they study the URL lexical features, like the one performed during this Master Thesis, and because we consider this kind of study better than to download and proccess the page (the thing that in fact is trying to be avoided). 

Hence, Kan and Thi \cite{Kan_URL05} focus their work in lexical features in order to classify as dangerous, URLs that were not previously in blacklists servers. They gather features like the URI components, length, ortographic data, or segments by entropy reduction. Their results are close to 95\% of accuracy.

On the other hand, this work not only focuses in lexical features of the requested URL, but also in other data that appears in the log files. And not exactly log data but Zhang et al. \cite{Zhang_cantina07}, with CANTINA, detect pishing URLs by studying lexical features, content related features, and a WHOIS query (obtaining the date when the domain were registered, which if it is too new, it can be less trustful). They also obtain a 95\% of accuracy.

The most important work we have found related to this were of J. Ma et al \cite{Ma_Url11}, whose aim is to detect malicious URLs, mainly related with pishing attacks throug e-mailing, but without processing the content or other private data of the user. They extract information from the lexical features (62\% of the total of the gathered features), and also from the host that has the URL. It is important to point out that they perform the study over 100 days, and that they work with a quantity of almost 2 million of features. In addition, they implement an online classifier (instead of a batch one), and obtain a 99\% of accuracy.

%Crime data mining and forensics in enterprise (?)

On the one hand, DM helped to develop new solutions to computer forensics \cite{DeVel2001}, being the researchers able to extract information from large files with events gathered from infected computers. Another important advance took place after the 9/11 events, when \textit{clustering techniques} and \textit{social network analysis} started to be performed in order to detect pontential crime networks \cite{Hsinchun2003}.
On the other hand, and more focused on the user side like our approach, there exist some user-centric solutions to problems like user authentication in a personal device, who Greenstadt and Beal \cite{cognitive_security_08} proposed to address using collected user biometrics along with machine learning techniques.

%feature extraction

%---------------------------------------------------------------------
%\subsection{Corporate Security}
%\label{subsec:corporatesecurity}

%---------------------------------------------------------------------
%\subsection{Machine Learning}
%\label{subsec:dm+ml}



%-------------------------------------------------------------------------------
%%%%%%%%%%%%%%%%%%%%%%%%%%%%%  PROBLEM DEFINITION  %%%%%%%%%%%%%%%%%%%%%%%%%%%%%
%-------------------------------------------------------------------------------

\section{Problem definition}
\label{sec:problem_data}

%*** Definir el problema a resolver y describir los datos que se van a manejar en el mismo.
Being introduced in Section \ref{sec:intro}, the problem this work addresses is the ability of self-adaptation that company network security systems have, for many new dangerous websites emerge every day. Corporate Security Policies (CSPs) may include Black and White lists to cope with dangerous situations, but always for known and already classified websites. Therefore, our goal is to demonstrate that is possible to automatically classify, as allowed or denied, a URL that was not in a blacklist or a whitelist. Also, we want that the final application of an `ALLOW' or `DENY' label depends on other features of that URL (lexical, or contextual, for instance), going then a step beyond the Black and White lists.

The data used to demonstrate this has been provided by an actual Spanish company and it has been anonymised. Since our purpose is to have a dataset with a set of connections of which it is already known its permission, we have used two different files:

\begin{itemize}
   \item A log file with the connections made by employees of the company. The data inside this log file were gathered along a period of two hours, from 8.30 to 10.30 am (30 minutes after the workday starts), monitoring the activity of all the employees in a medium-size Spanish company (80-100 people), obtaining 100000 patterns. We consider this dataset quite complete because it contains a very diverse amount of connection patterns, going from personal (traditionally addressed at the first hour of work) to professional issues (the rest of the day). This log contains the typical information of an HTTP request, it is gathered by the company Squid proxy \cite{squid:site}, and its format is CSV. 
   \item A set of rules wich decide, mostly depending on the URL core domain, if a connection should be allowed (whitelist) or denied (blacklist). The rules have been provided by the same company.
 \end{itemize} 

In order to have enough information to train the classifiers, we have taken as features the very same fields that the Squid log provides, being those of an HTTP request: the HTTP reply code and method, the time when the connection was made and how much did it take to the server to answer, the IP addresses of the client and requested server, the content type of the webpage and its size in bytes, and the complete URL string.
At the same time, the URL has the following structure:

\begin{verbatim}
http://www.subdomain5.....subdomain1.url_core.url_TLD/folder1/.../
/filename.filename_extension
\end{verbatim}

Thus a lot of features can be obtained from the URL string. Those can be lexical or non-lexical features. We name as lexical those features which are extracted from the characteristics of the URL string \cite{Kan_URL05}, such as URL string length, how many subdomains it has, how many letters or numbers are in the whole string, or if it contains folders or parameters. The non-lexical features of a URL can vary from the type of content of the website the user is accessing, to the properties of the host which contains the site \cite{Ma_Url11}. Here we have focused in lexical features, as most of the reviewed related works use this kind of features in their classification studies, and also for time reasons. When a new URL does not have an assigned class and we want to now if it should be allowed or denied, it much faster to analyse its lexical features than to gather its non-lexical properties.

It is important to point out that, from all the information which can be gathered from each entry of the log, not every field (or feature) is dependant on the user. Table \ref{featuretype} shows the set of initial features extracted from the log of connections. For each feature, its type is indicated and also the dependency with the user. This first study over the features can help to the feature selection, because what matters for identifying non desirable users behaviour, would be the information that depends on the user.

\begin{table*}[htpb]
\centering
 \caption{\label{featuretype} Extracted features from each entry of the Log and its dependance or independance on the user behaviour.}
{\scriptsize
\begin{tabular}{|l|l|l|}
\hline
\textbf{Feature name} & \textbf{Feature type} & \textbf{Relationship with user behaviour}\\ 
\hline
\texttt{http\_reply\_code} & Categorical & Independent \\ 
\texttt{http\_method} & Categorical & Independent \\ 
\texttt{duration\_milliseconds} & Numeric & Independent \\
\texttt{content\_type\_MCT} & Categorical & Independent \\ 
\texttt{content\_type} & Categorical & Independent \\ 
\texttt{server\_or\_cache\_address} & Categorical & Independent \\
\texttt{time} & Date & Dependent \\ 
\texttt{squid\_hierarchy} & Categorical & Independent \\ 
\texttt{bytes} & Numeric & Independent \\  
\texttt{URL\_length} & Numeric & Dependent \\  
\texttt{letters\_in\_URL} & Numeric & Dependent \\  
\texttt{digits\_in\_URL} & Numeric & Dependent \\  
\texttt{nonalphanumeric\_chars\_in\_URL} & Numeric & Dependent \\  
\texttt{url\_is\_IP} & Boolean & Dependent \\  
\texttt{url\_has\_subdomains} & Boolean & Dependent \\  
\texttt{num\_subdomains} & Numeric & Dependent \\  
\texttt{subdomain5} & Categorical & Dependent \\  
\texttt{subdomain4} & Categorical & Dependent \\  
\texttt{subdomain3} & Categorical & Dependent \\  
\texttt{subdomain2} & Categorical & Dependent \\  
\texttt{subdomain1} & Categorical & Dependent \\  
\texttt{url\_core} & Categorical & Dependent \\  
\texttt{url\_TLD} & Categorical & Dependent \\  
\texttt{url\_has\_path} & Boolean & Dependent \\  
\texttt{folder1} & Categorical & Dependent \\  
\texttt{folder2} & Categorical & Dependent \\  
\texttt{path\_has\_parameters} & Boolean & Dependent \\  
\texttt{num\_parameters} & Numeric & Dependent \\  
\texttt{url\_has\_file\_extension} & Boolean & Dependent \\  
\texttt{filename\_length} & Numeric & Dependent \\  
\texttt{letters\_in\_filename} & Numeric & Dependent \\  
\texttt{digits\_in\_filename} & Numeric & Dependent \\  
\texttt{other\_char\_in\_filename} & Numeric & Dependent \\  
\texttt{file\_extension} & Categorical & Dependent \\  
\texttt{url\_protocol} & Categorical & Dependent \\ 
\texttt{client\_address} & Categorical & Dependent \\
\hline
\end{tabular}
}
\end{table*}

Then, every original rule covers a number of entries, and so a class is applied to them. This class is a `label' with two possible values: `allow', or `deny'. An `allow' class is applied to an entry when that entry suffices the conditions of a rule which permits the connection, or because the URL is included in the whitelist. On the contrary, the `deny' class is assigned to those entries that are not permitted by the company, then, they will fit the rules that forbid those connections, or their URLs will be included in the blacklist.

Even though the whole methodology is detailed in the next Section, it is important to note now that in addition to the creation of a dataset with the labelled entries, it must be studied how to divide it in training and test datasets. For this reason, before we reach our main goal, we need to find a proper way to treat the initial dataset, and also to obtain a test setup that maximises the accuracy results and provides a good set of rules to demonstrate this work main purpose.

%----------------------------------------------------------------------------
%%%%%%%%%%%%%%%%%%%%%%%%%%%%%%%   METODOLOGY  %%%%%%%%%%%%%%%%%%%%%%%%%%%%%%%
%----------------------------------------------------------------------------

\section{Methodology}
\label{sec:methodology}

This Section is devoted to look over the steps that have been followed the realisation of different sets of experiments, in order to perform a deep study of the classifiers and data configurations. An experiment consists of obtaining a dataset configuration, a certain classifier, and trainining and test it to see the accuracy results.

The first and usual step \cite{Frank2011} is to preprocess all data and extract the information to have it in a way easy to work with. From the initial files, a connection log and a set of rules, we obtain a dataset with the data being processed and the rule decisions being applied. The nature of this dataset forces the application of some techniques for producing other datasets. In addition, the very results of the experiments justify the creation of new datasets, as will be explained in further subsections.

%*** Describir la metodolog�a a seguir para la creaci�n de los juegos de datos:
%- preprocesado
%- etiquetado para componer datos a clasificar
%- t�cnicas de balanceo

%*** Describir los 3 grandes bloques y justificarlos:
%- Datos iniciales
%- Caracter�sticas extra�das
%- Sesiones

All Java code for the realisation of this work is available at Github \cite{github:site}. % Short url http://git.io/4cQYFQ

%---------------------------------------------------------------------
\subsection{Initial Data}
\label{subsec:initial_data}

In order to have the extracted information in a proper way to train and test the classifiers, a preprocessing is performed over both log and rules files. From now on, we treat Black and White lists as set of rules too, given the fact that we can express every entry of those lists as:

\begin{verbatim}
IF
url = some_allowed_or_denied_url
THEN
allow/deny
\end{verbatim}

In a certain rule, the antecedents are expressed in the first place. They are the conditions that must be met in order to apply a decision. The antecedent for black and white lists and in the example is \textit{url = some\_allowed\_or\_denied\_url}, but there can be as much conditions as needed in a rule. Specialization of a rule is when it has a lot of conditions and covers a few patterns, while generalisation of rules means to have rules with less conditions but covering more patterns. %no estoy muy segura de esto �ltimo REVISAR
% Yo dir�a que es correcto, pero no he le�do el resto, no s� si viene
% a cuento - JJ

Our main goal is to be able to obtain rules with the ability to label connection patterns as allowed or denied, and that the conditions of those rules are not dependant on the URL. Thus, after the preprocessing and having extracted all the knowledge from the log and the rules, the entries in the Log can be labelled.

The initial log file has 100000 connection patterns. At the end of the labelling process, a dataset is formed with 58778 connection patterns, now with an `allow' or `deny' class assigned. The labelling process yields to 39044 allowed patterns and 19734 denied patterns. This dataset will be used for training and testing the classifiers, but before, is the one used for a first overview of the classifiers we intend to work with.

We have focused in two types of classifiers: rule based and tree based. This decision was based in the fact that 50\% of the features in every instance of the dataset is categorical. In order to measure the distance between two values of a categorical feature, one can use the Hamming or the Levenshtein distance \cite{chudnovsky2008method}, but this could work only for some features such as subdomains, for example. Furthermore, there are also numerical features. Then, we needed classifiers able to work with both numerical and categorical feature types.
Besides, they were chosen because its a set of rules what we aim to obtain. Also, the set of rules that these type of classifier generate from the training dataset helps with the feature selection.
This work has been developed in Java along with the Weka tool \cite{weka:site}, as it has several different classifiers of each type.

As can be seen in the ratio between allowed and denied patterns, the dataset is unbalanced and that can bias classification results \cite{maimon2005data}. There exist numerous techniques for balancing datasets, but here we have used two techniques that are mentioned in \cite{imbalance_techniques_02}. More specifically, these techniques are:

\begin{itemize}
   \item \textbf{Undersampling}. By this technique, the patterns of the majority class are reduced proportionally until both classes have the similar amount of patterns. This pattern removal has been performed randomly \cite{random_undersampling_08} and for that reason, three different balanced datasets have been created and tested, so that at the end the results have a mean and a standard deviation. Of course, removing patterns has a main disadvantage, which is the loss of information.
   \item \textbf{Oversampling}. On the contrary, with this method the patterns in the minority class are increased. Sometimes the difference between classes is filled with synthetic data \cite{smote_02}. When the features are numerical, is easier to introduce synthetic data similar to the extisting, because the differences between patterns can be numerically measured. In our case, most of the features are categorical so that is hard to create synthetic data. Then, the connection patterns of the minority class (allow) are doubled.
\end{itemize}

Then, having three datasets with the initial data, one unbalanced and two balanced with different methods, and 9 rule based plus 7 tree based classifiers, the datasets are tested by a 10-fold cross-validation process. This means that every dataset is divided in 10 equal parts and that the classifier will be trained an tested 10 times: each time it is trained with 9 of the 10 parts and the 10th is for testing \cite{Frank2011}. After the 10 generations, the results are calculated by averaging. This process is necessary to see classifiers' behaviour with the type of data we are working with. Also, Na�ve Bayes classifier \cite{Bayesian_Classifier_97} has been included as a reference for both its simplicity and good results with all data types. Table \ref{tabresults_todos} shows %something that I'll explain on monday

\begin{table}[htpb]
\centering 
{\small
\begin{tabular}{|l|c|c|}
\cline{2-3}
\multicolumn{1}{l|}{} & Undersampling & Oversampling \\ 
\hline
Na�ve Bayes & 91.02 $\pm$ 0.10 & 91.77 \\ 
\cline{1-1}
Conjunctive Rule & 60.00 $\pm$ 0.14 & 60.02 \\ 
\cline{1-1}
Decision Table & 94.31 $\pm$ 0.20 & 90.29 \\ 
\cline{1-1}
DTNB & 94.92 $\pm$ 0.14 & 95.65 \\ 
\cline{1-1}
JRip & 90.03 $\pm$ 0.07 & 92.47 \\ 
\cline{1-1}
NNge & \textbf{96.47} $\pm$ 0.02 & \textbf{98.76} \\ 
\cline{1-1}
One R & 93.53 $\pm$ 0.08 & 93.70 \\ 
\cline{1-1}
PART & \textbf{96.35} $\pm$ 0.09 & \textbf{97.54} \\ 
\cline{1-1}
Ridor & 86.60 $\pm$ 0.55 & 89.87 \\ 
\cline{1-1}
Zero R & 51.22 $\pm$ 0.16 & 51.26 \\ 
\cline{1-1}
AD Tree & 77.65 $\pm$ 0.08 & 77.68 \\ 
\cline{1-1}
Decision Stump & 60.00 $\pm$ 0.14 & 60.02 \\ 
\cline{1-1}
J48 & \textbf{96.99} $\pm$ 0.04 & \textbf{98.00} \\ 
\cline{1-1}
LAD Tree & 78.93 $\pm$ 1.71 & 79.97 \\ 
\cline{1-1}
Random Forest & \textbf{96.94} $\pm$ 0.07 & \textbf{98.84} \\ 
\cline{1-1}
Random Tree & 95.59 $\pm$ 0.40 & 98.35 \\ 
\cline{1-1}
REP Tree & \textbf{96.74} $\pm$ 0.04 & \textbf{97.67} \\ 
\hline
\end{tabular}
}
\caption[Global classification methods ranking. Classifiers are trained and tested by crossvalidation.]{\label{tabresults_todos} Results of all tested classification methods on balanced data. The best ones are marked in boldface.}
\end{table}


%*** Describir los primeros datos y procesamientos (eliminar duplicados, agrupar redundantes, etc)


%---------------------------------------------------------------------
\subsection{Extracting Features}
\label{subsec:extracting_features}

*** Describir y justificar las caracter�sticas extra�das de los datos iniciales.

%---------------------------------------------------------------------
%\subsection{Grouping in Sessions}
%\label{subsec:sessions}

%*** Describir y justificar el proceso de agrupamiento en sesiones

% Esto al final no entra en este art�culo



%----------------------------------------------------------------------------
%%%%%%%%%%%%%%%%%%%%%%%%%   EXPERIMENTS AND RESULTS  %%%%%%%%%%%%%%%%%%%%%%%%
%----------------------------------------------------------------------------

\section{Experiments and Results}
\label{sec:experiments}

*** Describir los experimentos realizados, en cada bloque.
*** Analizar los resultados obtenidos en cada bloque

%---------------------------------------------------------------------
\subsection{Initial Data Results}
\label{subsec:initial_data_results}

*** Experimentos y resultados sobre el conjunto original y los conjuntos 'refinados' (TLD, sin duplicados, etc)


%---------------------------------------------------------------------
\subsection{Extracted Features Results}
\label{subsec:extracted_features_results}

*** Experimentos y resultados sobre el conjunto con las nuevas caracter�sticas


%---------------------------------------------------------------------
\subsection{Sessions Results}
\label{subsec:sessions_results}

*** Experimentos y resultados sobre el conjunto agrupando por sesiones



%----------------------------------------------------------------------------
%%%%%%%%%%%%%%%%%%%%%%%%%%%%%%%   DISCUSSION  %%%%%%%%%%%%%%%%%%%%%%%%%%%%%%%
%----------------------------------------------------------------------------

\section{Discussion}
\label{sec:discussion}

*** Comentar los resultados de cada m�todo y analizar las ventajas e inconvenientes de cada uno, as� como las mejoras que se hayan conseguido con la extracci�n de caracter�sticas y el agrupamiento por sesiones


%----------------------------------------------------------------------------
%%%%%%%%%%%%%%%%%%%%%%%%%%%%%%%   CONCLUSIONS  %%%%%%%%%%%%%%%%%%%%%%%%%%%%%%%
%----------------------------------------------------------------------------

\section{Conclusions and Future Work}
\label{sec:conclusions}



%%%%%%%%%%%%%%%%%%%%%%%%%%%%%  ACKNOWLEDGEMENTS %%%%%%%%%%%%%%%%%%%%%%%%%%%%%%%%

\section{Acknowledgements}
This work has been supported by the European project MUSES (FP7-318508).

\bibliographystyle{elsarticle-num}
\bibliography{review_muses,data_mining_urls,ci_security_rules}

\end{document}
